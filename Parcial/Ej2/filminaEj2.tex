\documentclass[10pt]{beamer}

\usetheme{metropolis}
\usepackage{appendixnumberbeamer}

\usepackage[utf8]{inputenc}
\usepackage[spanish]{babel}

\usepackage{booktabs}

\usepackage{pgfplots}
\usepgfplotslibrary{dateplot}

\usepackage{xspace}

\usepackage{amssymb}
\usepackage{graphicx}
\usepackage{amsmath}


\newcounter{saveenumi}
\newcommand{\seti}{\setcounter{saveenumi}{\value{enumi}}}
\newcommand{\conti}{\setcounter{enumi}{\value{saveenumi}}}

\resetcounteronoverlays{saveenumi}


\newcommand{\Cfontc}{\fontsize{10.7}{9.5}\selectfont}
\newcommand{\Cfonta}{\fontsize{6.7}{9.5}\selectfont}
\newcommand{\Cfontb}{\fontsize{5.7}{9.5}\selectfont}
\newcommand{\Cfont}{\fontsize{5.5}{7.2}\selectfont}
\newcommand{\Cfonti}{\fontsize{8.5}{7.2}\selectfont}
\newcommand{\themename}{\textbf{\textsc{metropolis}}\xspace}
\newcommand{\powerset}{\mathcal{P}}
\newcommand{\R}{\rm I\!R}
\newcommand{\N}{\rm I\!N}
\newcommand{\A}{\mathbf{A}}
\newcommand{\poset}{(A, \leq^{\mathbf{A}})}
\newcommand{\Nand}{\wedge}

\newcommand{\refl}{\forall x\;x\leq x}
\newcommand{\Nrefl}{{A}_{\leq R}}
\newcommand{\anti}{\forall x\forall y\;\left( \left( x\leq y\wedge y\leq
x\right) \rightarrow x\equiv y\right)}
\newcommand{\Nanti}{{A}_{\leq A}}
\newcommand{\trans}{\forall x\forall y\forall z\;\left( (x\leq y\wedge
y\leq z)\rightarrow x\leq z\right)}
\newcommand{\Ntrans}{{A}_{\leq T}}
\newcommand{\sCot}{\forall x\forall y\;(x\leq x\;\mathsf{s}\ y\wedge y\leq x\;\mathsf{s}\;y)}
\newcommand{\NsCot}{{A}_{\mathsf{s}esC}}
\newcommand{\sLesCot}{\forall x\forall y\forall z\;\left( (x\leq
z\wedge y\leq z)\rightarrow x\;\text{$\mathsf{s\;}$}y\leq z\right)}
\newcommand{\NsLesCot}{{A}_{\mathsf{s}\leq C}}
\newcommand{\iCot}{\forall x\forall y\;(x\;\mathsf{i}\;y\leq
x\wedge x\;\mathsf{i}\;y\leq y)}
\newcommand{\NiCot}{{A}_{\mathsf{i}esC}}
\newcommand{\iGrCot}{\forall x\forall y\forall z\;\left( (z\leq
x\wedge z\leq y)\rightarrow z\leq x\;\mathsf{i}\;y\right)}
\newcommand{\NiGrCot}{{A}_{\mathsf{i}\geq C}}
\newcommand{\Tau}{\tau _{Ret}}
\newcommand{\Poset}{(A,\leq ^{\mathbf{A}})}
\newcommand{\assoc}{\forall x\forall y\forall
z\;(x\;\mathsf{s}$\ $y)\;\mathsf{s}\;z\leq x\;\mathsf{s}\;(y\;\mathsf{s}\;z)}
\newcommand{\acot}{(\forall xyz\;(x\;\mathsf{i}\
y)\;\mathsf{s}\;(x\;\mathsf{i}\;z)\equiv x\;\mathsf{i}\;(y\;\mathsf{s}%
\;z)\rightarrow \forall xy\left( \exists z(x\;\mathsf{i}\;z\equiv y\;\mathsf{%
i}\;z\wedge x\;\mathsf{s}\;z\equiv y\;\mathsf{s}\;z)\rightarrow x\equiv
y\right) )}
\newcommand{\idistr}{\forall xyz\;(x\;\mathsf{i}\
y)\;\mathsf{s}\;(x\;\mathsf{i}\;z)\equiv x\;\mathsf{i}\;(y\;\mathsf{s}%
\;z)}
\newcommand{\myconj}{x\;\mathsf{i}\;z\equiv y\;\mathsf{%
i}\;z\wedge x\;\mathsf{s}\;z\equiv y\;\mathsf{s}\;z}

\begin{document}

\begin{frame}{Teorema de Compacidad}

    \begin{itemize}[<+->]
      \item Sea $T$ una teor\'{\i}a tal que para cada $n\in \mathbf{N}$ hay un
          modelo $\mathbf{A}_{n}$ de $T$ tal que $\mathbf{A}_{n}$ tiene al menos $n$
          elementos. Probar que $T$ tiene un modelo infinito.
      \item Saque como corolario que
      si $\tau $ es un tipo cualquiera, entonces no hay una sentencia $\varphi \in
      S^{\tau }$ tal que para cada modelo de tipo $\tau $, se de que $\mathbf{A}%
      \vDash \varphi $ sii $A$ es finito. Analogamente, tampoco hay una sentencia $%
      \varphi \in S^{\tau }$ tal que para cada modelo de tipo $\tau $, se de que $%
      \mathbf{A}\vDash \varphi $ sii $A$ es infinito.
    \end{itemize}

\end{frame}


\begin{frame}{Teorema de Compacidad}
  Probemos ahora que
  \begin{center}
    Si $T$ una teor\'{\i}a tal que para cada $n\in \mathbf{N}$ hay un
    modelo $\mathbf{A}_{n}$ de $T$ tal que $\mathbf{A}_{n}$ tiene al menos $n$
    elementos entonces $T$ tiene un modelo infinito.
  \end{center}

\end{frame}



\begin{frame}{Teorema de Compacidad: $T$ tiene un modelo infinito}

  \begin{itemize}[<+->]
    \item Sea $\phi_{n}$ = $"$el universo del modelo tiene al menos $n$ elementos$"$
    \item Sea $\Sigma = \{\phi_{n} : n \in \mathbb{N}\}$
    \item \underline{Observación:} $\Sigma$ tiene infinitos elementos
    \item Sea $\tau$ un tipo
    \item Sea $T = (\Sigma, \tau)$ una teoría
  \end{itemize}

\end{frame}


\begin{frame}{Teorema de Compacidad: $T$ tiene un modelo infinito}

  \begin{itemize}[<+->]
    \item Por hipótesis, cada subconjunto finito de $\Sigma$ tiene un modelo
    \item Por Teorema de Compacidad sabemos que $T$ tiene un modelo
    \item Sea $\mathbb{A}$ dicho modelo
  \end{itemize}

\end{frame}


\begin{frame}{Teorema de Compacidad: $T$ tiene un modelo infinito}
  \begin{itemize}[<+->]
    \item Por definición de $\Sigma$:
      \begin{itemize}
        \item $n \leq |A|$, $\forall n \in \mathbb{N}$ donde $A$ es el universo de $\mathbb{A}$
      \end{itemize}
    \item Luego, $\mathbb{A}$ es infinito
  \end{itemize}

\end{frame}

\begin{frame}{No hay una sentencia $\varphi \in
S^{\tau }$ tal que para cada modelo de tipo $\tau $, se de que $\mathbf{A}%
\vDash \varphi $ sii $A$ es finito.}
  Ahora probemos por el absurdo que:
  \begin{center}
    Si $\tau $ es un tipo cualquiera, entonces no hay una sentencia $\varphi \in
    S^{\tau }$ tal que para cada modelo de tipo $\tau $, se de que $\mathbf{A}%
    \vDash \varphi $ sii $A$ es finito. Analogamente, tampoco hay una sentencia $%
    \varphi \in S^{\tau }$ tal que para cada modelo de tipo $\tau $, se de que $%
    \mathbf{A}\vDash \varphi $ sii $A$ es infinito.

  \end{center}

\end{frame}


\begin{frame}{No hay una sentencia $\varphi \in
S^{\tau }$ tal que para cada modelo de tipo $\tau $, se de que $\mathbf{A}%
\vDash \varphi $ sii $A$ es finito.}
  \begin{itemize}[<+->]
    \item Sea $\tau$ un tipo cualquiera
    \item Sea $\Sigma \subseteq S^{\tau}$ tal que $\mathbb{A} \vDash \Sigma$
    si y sólo si $\mathbb{A}$ es finito
    \item Luego, por la propiedad del item anterior, la teoría $(\Sigma, \tau)$ tiene
    modelos finitos arbitrariamente grandes
  \end{itemize}
\end{frame}

\begin{frame}{No hay una sentencia $\varphi \in
S^{\tau }$ tal que para cada modelo de tipo $\tau $, se de que $\mathbf{A}%
\vDash \varphi $ sii $A$ es finito.}
  \begin{itemize}
    \item Así, por el lema probado anteriormente, $(\Sigma, \tau)$ tiene un modelo infinito
  \end{itemize}
  \pause
  \begin{center}
    \Cfontc $ABSURDO$
  \end{center}
\end{frame}

\end{document}
