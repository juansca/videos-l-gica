
\documentclass{article}
%%%%%%%%%%%%%%%%%%%%%%%%%%%%%%%%%%%%%%%%%%%%%%%%%%%%%%%%%%%%%%%%%%%%%%%%%%%%%%%%%%%%%%%%%%%%%%%%%%%%%%%%%%%%%%%%%%%%%%%%%%%%%%%%%%%%%%%%%%%%%%%%%%%%%%%%%%%%%%%%%%%%%%%%%%%%%%%%%%%%%%%%%%%%%%%%%%%%%%%%%%%%%%%%%%%%%%%%%%%%%%%%%%%%%%%%%%%%%%%%%%%%%%%%%%%%
\usepackage{amssymb}
\usepackage{graphicx}
\usepackage{amsmath}
\usepackage[a4paper, margin=0.9in]{geometry}

\setcounter{MaxMatrixCols}{10}
%TCIDATA{OutputFilter=LATEX.DLL}
%TCIDATA{Version=5.50.0.2953}
%TCIDATA{<META NAME="SaveForMode" CONTENT="1">}
%TCIDATA{BibliographyScheme=Manual}
%TCIDATA{Created=Mon Dec 09 16:16:07 2002}
%TCIDATA{LastRevised=Saturday, November 04, 2017 08:11:05}
%TCIDATA{<META NAME="GraphicsSave" CONTENT="32">}
%TCIDATA{<META NAME="DocumentShell" CONTENT="General\Blank Document">}
%TCIDATA{CSTFile=LaTeX article (bright).cst}

\newtheorem{theorem}{Theorem}
\newtheorem{acknowledgement}[theorem]{Acknowledgement}
\newtheorem{algorithm}[theorem]{Algorithm}
\newtheorem{axiom}[theorem]{Axiom}
\newtheorem{case}[theorem]{Case}
\newtheorem{claim}[theorem]{Claim}
\newtheorem{conclusion}[theorem]{Conclusion}
\newtheorem{condition}[theorem]{Condition}
\newtheorem{conjecture}[theorem]{Conjecture}
\newtheorem{corollary}[theorem]{Corollary}
\newtheorem{criterion}[theorem]{Criterion}
\newtheorem{definition}[theorem]{Definition}
\newtheorem{example}[theorem]{Example}
\newtheorem{exercise}[theorem]{Exercise}
\newtheorem{lemma}[theorem]{Lemma}
\newtheorem{notation}[theorem]{Notation}
\newtheorem{problem}[theorem]{Problem}
\newtheorem{proposition}[theorem]{Proposition}
\newtheorem{remark}[theorem]{Remark}
\newtheorem{solution}[theorem]{Solution}
\newtheorem{summary}[theorem]{Summary}
\newenvironment{proof}[1][Proof]{\textbf{#1.} }{\ \rule{0.5em}{0.5em}}
\input{tcilatex}
\begin{document}


\begin{enumerate}
\item Sea $\tau _{Ret}=(\emptyset ,\{\mathsf{s}^{2},\mathsf{i}^{2}\},\{\leq
^{2}\},a)$. Y sea $Ret=(\Sigma _{Ret},\tau _{Ret})$, donde $\Sigma _{Ret}$
es el siguiente conjunto de sentencias:%
\begin{eqnarray*}
\mathrm{A}_{\leq R} &=&\forall x\;x\leq x \\
\mathrm{A}_{\leq A} &=&\forall x\forall y\;\left( \left( x\leq y\wedge y\leq
x\right) \rightarrow x\equiv y\right)  \\
\mathrm{A}_{\leq T} &=&\forall x\forall y\forall z\;\left( (x\leq y\wedge
y\leq z)\rightarrow x\leq z\right)  \\
\mathrm{A}_{\mathsf{s}esC} &=&\forall x\forall y\;(x\leq x\;\mathsf{s}%
\;y\wedge y\leq x\;\mathsf{s}\;y) \\
\mathrm{A}_{\mathsf{s}\leq C} &=&\forall x\forall y\forall z\;\left( (x\leq
z\wedge y\leq z)\rightarrow x\;\text{$\mathsf{s\;}$}y\leq z\right)  \\
\mathrm{A}_{\mathsf{i}esC} &=&\forall x\forall y\;(x\;\mathsf{i}\;y\leq
x\wedge x\;\mathsf{i}\;y\leq y) \\
\mathrm{A}_{\mathsf{i}\geq C} &=&\forall x\forall y\forall z\;\left( (z\leq
x\wedge z\leq y)\rightarrow z\leq x\;\mathsf{i}\;y\right) 
\end{eqnarray*}

\begin{enumerate}
\item Note que dada una estructura $\mathbf{A}$\ de tipo $\tau _{Ret}$, se
tiene que $\mathbf{A}$ satisface los axiomas $\mathrm{A}_{\leq R}$, $\mathrm{%
A}_{\leq A}$ y $\mathrm{A}_{\leq T}$ sii $(A,\leq ^{\mathbf{A}})$ es un
poset.

\item Describa las estructuras que satisfacen las sentencias $\mathrm{A}%
_{\leq R}$, $\mathrm{A}_{\leq A}$, $\mathrm{A}_{\leq T}$ y $\mathrm{A}_{%
\mathsf{s}esC}$

\item Describa las estructuras que satisfacen las sentencias $\mathrm{A}%
_{\leq R}$, $\mathrm{A}_{\leq A}$, $\mathrm{A}_{\leq T}$ y $\mathrm{A}_{%
\mathsf{s}\leq C}$

\item Describa las estructuras que son modelos de $Ret$

\item De una prueba en $Ret$ de la sentencia $\forall x\forall y\forall
z\;(x\;\mathsf{s}$\ $y)\;\mathsf{s}\;z\leq x\;\mathsf{s}\;(y\;\mathsf{s}\;z)$

\item De una prueba en $Ret$ de la sentencia $(\forall xyz\;(x\;\mathsf{i}\
y)\;\mathsf{s}\;(x\;\mathsf{i}\;z)\equiv x\;\mathsf{i}\;(y\;\mathsf{s}%
\;z)\rightarrow \forall xy\left( \exists z(x\;\mathsf{i}\;z\equiv y\;\mathsf{%
i}\;z\wedge x\;\mathsf{s}\;z\equiv y\;\mathsf{s}\;z)\rightarrow x\equiv
y\right) )$
\end{enumerate}

\item Sea $T$ una teor\'{\i}a tal que para cada $n\in \mathbf{N}$ hay un
modelo $\mathbf{A}_{n}$ de $T$ tal que $\mathbf{A}_{n}$ tiene al menos $n$
elementos (o sea una teor\'{\i}a que tiene modelos finitos tan grandes se
quiera). Probar que $T$ tiene un modelo infinito. Saque como corolario que
si $\tau $ es un tipo cualquiera, entonces no hay una sentencia $\varphi \in
S^{\tau }$ tal que para cada modelo de tipo $\tau $, se de que $\mathbf{A}%
\vDash \varphi $ sii $A$ es finito. Analogamente, tampoco hay una sentencia $%
\varphi \in S^{\tau }$ tal que para cada modelo de tipo $\tau $, se de que $%
\mathbf{A}\vDash \varphi $ sii $A$ es infinito.
\end{enumerate}

\end{document}
