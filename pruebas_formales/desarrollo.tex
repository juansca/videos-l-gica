\documentclass[10pt]{beamer}

\usetheme{metropolis}
\usepackage{appendixnumberbeamer}

\usepackage[utf8]{inputenc}
\usepackage[spanish]{babel}

\usepackage{booktabs}

\usepackage{pgfplots}
\usepgfplotslibrary{dateplot}

\usepackage{xspace}

\usepackage{amssymb}
\usepackage{graphicx}
\usepackage{amsmath}


\newcounter{saveenumi}
\newcommand{\seti}{\setcounter{saveenumi}{\value{enumi}}}
\newcommand{\conti}{\setcounter{enumi}{\value{saveenumi}}}

\resetcounteronoverlays{saveenumi}


\newcommand{\Cfonta}{\fontsize{6.7}{9.5}\selectfont}
\newcommand{\Cfontb}{\fontsize{5.7}{9.5}\selectfont}
\newcommand{\Cfont}{\fontsize{5.5}{7.2}\selectfont}
\newcommand{\Cfonti}{\fontsize{8.5}{7.2}\selectfont}
\newcommand{\themename}{\textbf{\textsc{metropolis}}\xspace}
\newcommand{\powerset}{\mathcal{P}}
\newcommand{\R}{\rm I\!R}
\newcommand{\N}{\rm I\!N}
\newcommand{\A}{\mathbf{A}}
\newcommand{\poset}{(A, \leq^{\mathbf{A}})}
\newcommand{\Nand}{\wedge}

\newcommand{\refl}{\forall x\;x\leq x}
\newcommand{\Nrefl}{{A}_{\leq R}}
\newcommand{\anti}{\forall x\forall y\;\left( \left( x\leq y\wedge y\leq
x\right) \rightarrow x\equiv y\right)}
\newcommand{\Nanti}{{A}_{\leq A}}
\newcommand{\trans}{\forall x\forall y\forall z\;\left( (x\leq y\wedge
y\leq z)\rightarrow x\leq z\right)}
\newcommand{\Ntrans}{{A}_{\leq T}}
\newcommand{\sCot}{\forall x\forall y\;(x\leq x\;\mathsf{s}\ y\wedge y\leq x\;\mathsf{s}\;y)}
\newcommand{\NsCot}{{A}_{\mathsf{s}esC}}
\newcommand{\sLesCot}{\forall x\forall y\forall z\;\left( (x\leq
z\wedge y\leq z)\rightarrow x\;\text{$\mathsf{s\;}$}y\leq z\right)}
\newcommand{\NsLesCot}{{A}_{\mathsf{s}\leq C}}
\newcommand{\iCot}{\forall x\forall y\;(x\;\mathsf{i}\;y\leq
x\wedge x\;\mathsf{i}\;y\leq y)}
\newcommand{\NiCot}{{A}_{\mathsf{i}esC}}
\newcommand{\iGrCot}{\forall x\forall y\forall z\;\left( (z\leq
x\wedge z\leq y)\rightarrow z\leq x\;\mathsf{i}\;y\right)}
\newcommand{\NiGrCot}{{A}_{\mathsf{i}\geq C}}
\newcommand{\Tau}{\tau _{Ret}}
\newcommand{\Poset}{(A,\leq ^{\mathbf{A}})}
\newcommand{\assoc}{\forall x\forall y\forall
z\;(x\;\mathsf{s}$\ $y)\;\mathsf{s}\;z\leq x\;\mathsf{s}\;(y\;\mathsf{s}\;z)}
\newcommand{\acot}{Dis1\rightarrow CancDobl}

\newcommand{\idistr}{\forall xyz\;(x\;\mathsf{i}\
y)\;\mathsf{s}\;(x\;\mathsf{i}\;z)\equiv x\;\mathsf{i}\;(y\;\mathsf{s}%
\;z)}
\newcommand{\myconj}{x\;\mathsf{i}\;z\equiv y\;\mathsf{%
i}\;z\wedge x\;\mathsf{s}\;z\equiv y\;\mathsf{s}\;z}

\begin{document}

\begin{frame}{Ret}
  \begin{center}

    Sea $\tau _{Ret}=(\emptyset ,\{\mathsf{s}^{2},\mathsf{i}^{2}\},\{\leq
    ^{2}\},a)$. Y sea $Ret=(\Sigma _{Ret},\tau _{Ret})$, donde $\Sigma _{Ret}$
    es el siguiente conjunto de sentencias:
    \pause
    \begin{align}
       \Nrefl &= \refl \nonumber\\
       \Nanti &= \anti \nonumber \\
       \Ntrans &= \trans \nonumber \\
       \NsCot &= \sCot \nonumber \\
       \NsLesCot &= \sLesCot \nonumber \\
       \NiCot &= \iCot \nonumber \\
       \NiGrCot &= \iGrCot \nonumber
    \end{align}
    \setcounter{equation}{0}

    \stepcounter{beamerpauses}


  \end{center}
\end{frame}

\begin{frame}{Ret}
  Daremos pruebas formales para las siguientes sentencias:
  \stepcounter{beamerpauses}
  \begin{itemize}[<+->]

  \item $ \phi = \assoc $
  \item  $ \psi = \acot $
  \\ Donde \\
    \begin{align}
    Dis1 &= \forall xyz\;(x\;\mathsf{i}\
    y)\;\mathsf{s}\;(x\;\mathsf{i}\;z)\equiv x\;\mathsf{i}\;(y\;\mathsf{s}%
    \;z) \nonumber \\
    CancDobl &= \forall xy\left( \exists z(x\;\mathsf{i}\;z\equiv y\;\mathsf{%
    i}\;z\wedge x\;\mathsf{s}\;z\equiv y\;\mathsf{s}\;z)\rightarrow x\equiv
    y\right) \nonumber
  \end{align}
  \end{itemize}

\end{frame}


\begin{frame}{Modelos de Ret: $ \Nrefl, \Nanti, \Ntrans$}

  %% Comenzaremos estudiando qué significa cada axioma de Ret:
  \begin{itemize}[<+->]

    \item $ \A \models \Nrefl \Nand \Nanti \Nand \Ntrans $ si y sólo si
    $\poset$ es un poset

    \item \textbf{Observación:} Una estructura $\A$ de tipo $\tau_{Ret}$ puede
            satisfacer los 3 axiomas pero esto no significa que las operaciones
            $ s^{\A} $ e $ i^{\A} $ sean las operaciones supremo e ínfimo respecto
            al orden $ \leq^{\A} $.
  \end{itemize}
\end{frame}


\begin{frame}{Modelos de Ret: $\Nrefl, \Nanti, \Ntrans, \NsCot, \NsLesCot$}
  Si $ \A \models \Nrefl \Nand \Nanti \Nand \Ntrans $ entonces:
  \begin{itemize}[<+->]

    \item $\A \models \NsCot$ si y sólo si $(a\ s^{\A}\ b)$ es cota superior
    de $\{a,b\}$ en $\poset$ cualesquiera sean $a$ y $b$.
    \item $\A \models \NsLesCot$ si y sólo si $(a\ s^{\A}\ b)$
    es menor o igual a toda cota superior de $\{a,b\}$ cualesquiera sean $a$ y $b$
    \item $\A$ cumplirá los axiomas
    $\Nrefl, \Nanti, \Ntrans, \NsCot\ y\ \NsLesCot$ si y sólo si $\leq^{\A}$
    es un orden parcial y $s^{\A}$ es la operación supremo respecto del orden
    $\leq^{\A}$.
  \end{itemize}
\end{frame}


\begin{frame}{Modelos de Ret}
  $\A$ es un modelo de $Ret$ si y sólo si se cumple que:
  \begin{itemize}[<+->]
    \pause
    \item $\poset$ es un orden parcial
    \item $s^{\A}$ es el supremo en el poset $\poset$
    \item $i^{\A}$ es el ínfimo en el poset $\poset$
  \end{itemize}
\end{frame}


\begin{frame}{Estructura de Prueba}
  Encontraremos una prueba formal en Ret de la sentencia:
  \begin{center}
    $\assoc$
  \end{center}
  \pause

  Forma de encontrar la prueba formal:
  \pause
  \begin{enumerate}[<+->]

    \item Prueba Matemática
    \item Prueba Formal
  \end{enumerate}
\end{frame}


\begin{frame}{$\assoc$: Prueba Matemática}
  \begin{itemize}[<+->]

    \item Sean $a, b, c$ elementos de A fijos.
    \item Probaremos que
      \begin{center}
        $ (a\ s\ b)\ s\ c  \leq a\ s\ (b\ s\ c) $
      \end{center}
    \item Sabemos por $\NsCot$ que
      \begin{align}
        &a \leq a\ s\ (b\ s\ c) \label{eq1} \\
        &b\ s\ c \leq a\ s\ (b\ s\ c) \label{eq2}
      \end{align}
    \item Aplicandolo nuevamente, sabemos que
      \begin{align}
        &b \leq (b\ s\ c) \label{eq3}\\
        &c \leq (b\ s\ c) \label{eq4}
      \end{align}

  \end{itemize}

\end{frame}

\begin{frame}{$\assoc$: Prueba Matemática}
  \begin{itemize}[<+->]

    \item Luego, por \eqref{eq2}, \eqref{eq3} y $\Ntrans$ tenemos
      \begin{align}
        &b \leq a\ s\ (b\ s\ c) \label{eq5}
      \end{align}

    \item Y por \eqref{eq2}, \eqref{eq4} y $\Ntrans$ tenemos
    \begin{align}
      &c \leq a\ s\ (b\ s\ c) \label{eq6}
    \end{align}

    \item Es decir, hasta aquí hemos probado que

      \begin{align}
        &a \leq a\ s\ (b\ s\ c) \label{eq1} \\
        &b \leq a\ s\ (b\ s\ c) \label{eq5} \\
        &c \leq a\ s\ (b\ s\ c) \label{eq6}
      \end{align}

  \end{itemize}
\end{frame}


\begin{frame}{$\assoc$: Prueba Matemática}
  \begin{itemize}[<+->]
    \item Por $\NsLesCot$, tomando
      \begin{center}
        \begin{align}
          x &= a \nonumber \\
          y &= b \nonumber \\
          z &= a\ s\ (b\ s\ c) \nonumber
        \end{align}
      \end{center}
      tenemos que
        \begin{align}
          &a\ s\ b  \leq a\ s\ (b\ s\ c) \label{eq7}
        \end{align}

  \end{itemize}
\end{frame}




\begin{frame}{$\assoc$: Prueba Matemática}
  \begin{itemize}[<+->]
    \item Finalmente, si aplicamos nuevamente $\NsLesCot$ tomando
    \begin{center}
      \begin{align}
        x &= a\ s\ b  \nonumber \\
        y &= c \nonumber \\
        z &= a\ s\ (b\ s\ c) \nonumber
      \end{align}

    \end{center}

     obtenemos
      \begin{align}
        &(a\ s\ b)\ s\ c  \leq a\ s\ (b\ s\ c) \label{eq7}
      \end{align}
    \item Como $a, b, c$ eran elementos cualesquiera, probamos que
      \begin{center}
        $\assoc$
      \end{center}

  \end{itemize}


\end{frame}



\begin{frame}{$\assoc$: Prueba Formal}
  Ahora daremos la prueba formal en $Ret$ de la sentencia en cuestión
\end{frame}

\begin{frame}{$\assoc$: Prueba Formal}
% New command to change font size
\Cfonta
  \begin{enumerate}[<+->]
    \item $ \sCot $ \hfill AXIOMAPROPIO
    \item $ a \leq a\ s\ (b\ s\ c)\Nand (b\ s\ c)\leq a\ s\ (b\ s\ c) $ \hfill PARTICULARIZACIONx2(1)
    \item $ b\ s\ c \leq a\ s\ (b\ s\ c)$ \hfill CONJELIM(2)
    \item $ b \leq b\ s\ c \Nand c \leq b\ s\ c $ \hfill PARTICULARIZACIONx2(1)
    \item $ b \leq (b\ s\ c)$ \hfill CONJELIM(4)
    \item $ b \leq (b\ s\ c) \Nand b\ s\ c \leq a\ s\ (b\ s\ c)$ \hfill CONJINT(5,3)
    \item $ \trans $ \hfill AXIOMAPROPIO
    \item $ (b \leq (b\ s\ c) \Nand (b\ s\ c) \leq a\ s\ (b\ s\ c)) \rightarrow (b \leq a\ s\ (b\ s\ c))$ \hfill PARTICULARIZACIONx3(7)
    \item $ b \leq a\ s\ (b\ s\ c) $ \hfill MODUSPONENS(6,8)
    \item $ c \leq (b\ s\ c)$ \hfill CONJELIM(4)
    \item $ c \leq (b\ s\ c) \Nand  b\ s\ c \leq a\ s\ (b\ s\ c) $ \hfill CONJINT(10,3)
    \item $ (c \leq (b\ s\ c) \Nand (b\ s\ c) \leq a\ s\ (b\ s\ c)) \rightarrow (c \leq a\ s\ (b\ s\ c))$ \hfill PARTICULARIZACIONx3(7)
    \item $ c \leq a\ s\ (b\ s\ c) $ \hfill MODUSPONENS(11,12)
    \seti
  \end{enumerate}

    %% Cortar acá!

  \setcounter{equation}{0}

\end{frame}


\begin{frame}{$\assoc$: Prueba Formal}
% New command to change font size
\Cfontb

\begin{enumerate}[<+->]
  \conti
  \item $ a \leq a\ s\ (b\ s\ c)$ \hfill CONJELIM(2)

  \item $ a \leq a\ s\ (b\ s\ c) \Nand b \leq a\ s\ (b\ s\ c)$ \hfill CONJINT(14,9)
  \item $ \sLesCot $ \hfill AXIOMAPROPIO
  \item $ (a \leq a\ s\ (b\ s\ c) \Nand b \leq a\ s\ (b\ s\ c)) \rightarrow ((a\ s\ b) \leq a\ s\ (b\ s\ c)) $ \hfill PARTICULARIZACIONx3(16)
  \item $ (a\ s\ b) \leq a\ s\ (b\ s\ c)$ \hfill MODUSPONENS(15,17)

  \item $ ((a\ s\ b) \leq a\ s\ (b\ s\ c) \Nand c \leq a\ s\ (b\ s\ c)) \rightarrow ((a\ s\ b)\ s\ c \leq a\ s\ (b\ s\ c)) $ \hfill PARTICULARIZACIONx3(16)
  \item $ (a\ s\ b) \leq a\ s\ (b\ s\ c) \Nand c \leq a\ s\ (b\ s\ c) $ \hfill CONJINT(18,13)
  \item $ (a\ s\ b)\ s\ c \leq a\ s\ (b\ s\ c) $ \hfill MODUSPONENS(20,19)
  \item $ \forall x \forall y \forall z (x\ s\ y)\ s\ z \leq x\ s\ (y\ s\ z)$ \hfill GENERALIZACIÓNx3(21)
  \item $ \forall x \forall y \forall z (x\ s\ y)\ s\ z \leq x\ s\ (y\ s\ z)$ \hfill CONCLUSION(22)
  \seti
\end{enumerate}

\end{frame}



\begin{frame}{\Cfonti$\idistr \rightarrow \forall x\ \forall y\ (\exists z (\myconj) \rightarrow x\ \equiv y)$:}

  Ahora encontraremos una prueba formal en Ret de la sentencia $\psi$:
  \begin{center}
    $\acot$
  \end{center}
  \pause
  Utilizaremos la misma forma de encontrar la prueba formal usada para probar
  $\phi$

\end{frame}


\begin{frame}{\Cfonti$\idistr \rightarrow \forall x\ \forall y\ (\exists z (\myconj) \rightarrow x\ \equiv y)$: Prueba Matemática}

  Para probar $\psi$, será útil un teorema auxiliar:
  \begin{center}
    $\forall x \forall y ((x\ s\ y)\ i\ x \equiv x)$
  \end{center}
  A partir de ahora, lo llamaremos TeoremaAbsorv. Queda como ejercicio su prueba.

\end{frame}



\begin{frame}{\Cfonti$\idistr \rightarrow \forall x\ \forall y\ (\exists z (\myconj) \rightarrow x\ \equiv y)$:
   Prueba Matemática}
  Ahora si probaremos
  \begin{center}
    $\acot$
  \end{center}
  \pause
  \begin{itemize}[<+->]

    \item Primero, supongamos que se cumple que
    \begin{align}
      \forall x \forall y \forall z (x\ i\ y) s\ (x\ i\ z) \equiv x\ i\ (y\ s\ z) \label{eq11}
    \end{align}
    \item Probaremos que
    \begin{center}
      $\forall x \forall y (\exists z ((x\ i\ z \equiv y\ i\ z \wedge x\ s\ z \equiv y\ s\ z) \rightarrow x \equiv y))$
    \end{center}

  \end{itemize}
\end{frame}


\begin{frame}{\Cfonti$\idistr \rightarrow \forall x\ \forall y\ (\exists z (\myconj) \rightarrow x\ \equiv y)$:
   Prueba Matemática}
  \begin{itemize}[<+->]

    \item Sean $a, b$ dos elementos de A fijos
    \item Probaremos que
    \begin{center}
      $\exists z (a\ i\ z \equiv b\ i\ z \wedge a\ s\ z \equiv b\ s\ z) \rightarrow a \equiv b$
    \end{center}
    \item Supongamos que:
    \begin{center}
      $\exists z (a\ i\ z \equiv b\ i\ z \wedge a\ s\ z \equiv b\ s\ z)$
    \end{center}

    \item Supongamos $c$ un elemento que cumple
    \begin{align}
      (a\ i\ c) \equiv (b\ i\ c) \wedge (a\ s\ c) \equiv (b\ s\ c)
    \end{align}
    \item Probaremos que
    \begin{center}
      $a \equiv b$
    \end{center}
  \end{itemize}
\end{frame}


\begin{frame}{\Cfonti$\idistr \rightarrow \forall x\ \forall y\ (\exists z (\myconj) \rightarrow x\ \equiv y)$:
   Prueba Matemática}
  \begin{itemize}[<+->]

    \item Por TeoremaAbsorv sabemos que
      \begin{align}
        (b\ s\ c)\ i\ b \equiv b
      \end{align}
    \item Por (2):
      \begin{align}
        (b\ s\ c)\ i\ b \equiv (a\ s\ c)\ i\ b
      \end{align}
    \item Por (1):
      \begin{align}
        (a\ s\ c)\ i\ b \equiv (a\ i\ b)\ s\ (c\ i\ b)
      \end{align}
    \item Nuevamente por (2):
      \begin{align}
        (a\ i\ b)\ s\ (c\ i\ b) \equiv (a\ i\ b)\ s\ (c\ i\ a)
      \end{align}

    \item Nuevamente por (1) y por conmutatividad de $i$:
      \begin{align}
        (a\ i\ b)\ s\ (c\ i\ a) \equiv (b\ s\ c)\ i\ a
      \end{align}


  \end{itemize}
\end{frame}


\begin{frame}{\Cfonti$\idistr \rightarrow \forall x\ \forall y\ (\exists z (\myconj) \rightarrow x\ \equiv y)$:
   Prueba Matemática}
  \begin{itemize}[<+->]
    \item Y así, por (2):
      \begin{align}
        (b\ s\ c)\ i\ a \equiv (a\ s\ c)\ i\ a
      \end{align}

    \item Finalmente, por TeoremaAbsorv:
      \begin{align}
        (a\ s\ c)\ i\ a \equiv a
      \end{align}
    \item Es decir,
      \begin{center}
        $a \equiv b$
      \end{center}

    \item Por lo tanto, se cumple que
      \begin{center}
        $\acot$
      \end{center}

  \end{itemize}
\end{frame}

\begin{frame}{\Cfonti$\idistr \rightarrow \forall x\ \forall y\ (\exists z (\myconj) \rightarrow x\ \equiv y)$:
   Prueba Formal}
  \begin{center}
    Ahora daremos la prueba formal en Ret de la sentencia en cuestión
    \pause
  \end{center}
  Utilizaremos los siguientes teoremas cuyas pruebas formales son dejadas al lector
  \begin{itemize}[<+->]
    \item $\forall x\ \forall y\ (x\ s\ y)\ i\ x \equiv x$  (TeoremaAbsorv)
    \item $\forall x\ \forall y\ (x\ i\ y) \equiv (y\ i\ x)$  (TeoremaConmut)
  \end{itemize}
\end{frame}


\begin{frame}{\Cfonti$\idistr \rightarrow \forall x\ \forall y\ (\exists z (\myconj) \rightarrow x\ \equiv y)$:
    Prueba Formal
}
% New command to change font size
\Cfont
  \begin{enumerate}[<+->]
    \item $ \idistr $ \hfill HIPÓTESIS1
    %\item $ \forall x\ \forall y\ (\exists z (\myconj) \rightarrow x\ \equiv y)$ \hfill(HIPÓTESIS2)
    \item $ \exists z\ ((a\ i\ z) \equiv (b\ i\ z) \Nand (a\ s\ z) \equiv (b\ s\ z))$ \hfill HIPÓTESIS2
    \item $ ((a\ i\ c) \equiv (b\ i\ c) \Nand (a\ s\ c) \equiv (b\ s\ c))$ \hfill ELECCION(2)
    \item $ \forall x\ \forall y\ (x\ s\ y)\ i\ x \equiv x$ \hfill TEOREMAABSORV
    \item $ (b\ s\ c)\ i\ b \equiv b$ \hfill PARTICULARIZACIONx2(4)
    \item $ (a\ s\ c) \equiv (b\ s\ c) $ \hfill CONJELIM(3)

    \item $ (a\ s\ c)\ i\ b \equiv b$ \hfill REEMP(6,5)
    \item $ (b\ i\ a)\ s\ (b\ i\ c) \equiv b\ i\ (a\ s\ c) $ \hfill PARTICULARIZACIONx3(1)
    \item $ b\ i\ (a\ s\ c) \equiv b$ \hfill TEOREMACONMUT(7)
    \item $ (b\ i\ a)\ s\ (b\ i\ c) \equiv b$ \hfill REEMP(8,9)
    \item $ (a\ i\ c) \equiv (b\ i\ c) $ \hfill CONJELIM(3)

    \item $ (b\ i\ a)\ s\ (a\ i\ c) \equiv b$ \hfill REEMP(10,11)
    \item $ (a\ i\ b)\ s\ (a\ i\ c) \equiv b$ \hfill TEOREMACONMUT(12)
    \item $ (a\ i\ b)\ s\ (a\ i\ c) \equiv a\ i\ (b\ s\ c)$ \hfill PARTICULARIZACIONx3(1)
    \item $ a\ i\ (b\ s\ c) \equiv b $ \hfill REEMP(14,13)
    \item $ a\ i\ (a\ s\ c) \equiv b $ \hfill REEMP(6,15)
    \item $ (a\ s\ c)\ i\ a \equiv b$ \hfill TEOREMACONMUT(16)
    \item $ (a\ s\ c)\ i\ a \equiv a$ \hfill PARTICULARIZACIONx2(4)
    \item $ a \equiv b$ \hfill TESIS2REEMP(18,17)
    \item $ \exists z\ ((a\ i\ z) \equiv (b\ i\ z) \Nand (a\ s\ z) \equiv (b\ i\ z)) \rightarrow a\ \equiv b$ \hfill CONCLUSION
    \item $ \forall x\ \forall y\ (\exists z (\myconj) \rightarrow x\ \equiv y)$ \hfill TESIS1GENERALIZACIÓNx2(20)
    \item $ \idistr \rightarrow \forall x\ \forall y\ (\exists z (\myconj) \rightarrow x\ \equiv y)$ \hfill CONCLUSION
    %% Cortar acá!

  \end{enumerate}
  \setcounter{equation}{0}

\end{frame}


%\Nrefl, \Nanti, \Ntrans, \NsCot\ y\ \NsLesCot$


%\begin{frame}{Absurdo}
%  \begin{itemize}[<+->]
%
%    \item Vimos que $a$ $X$ y que $X$ $s$.
%  \end{itemize}
%\end{frame}

\end{document}
