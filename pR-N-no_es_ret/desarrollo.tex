\documentclass[10pt]{beamer}

\usetheme{metropolis}
\usepackage{appendixnumberbeamer}

\usepackage[utf8]{inputenc}
\usepackage[spanish]{babel}

\usepackage{booktabs}
\usepackage[scale=2]{ccicons}

\usepackage{pgfplots}
\usepgfplotslibrary{dateplot}

\usepackage{xspace}
\newcommand{\themename}{\textbf{\textsc{metropolis}}\xspace}
\newcommand{\powerset}{\mathcal{P}}
\newcommand{\R}{\rm I\!R}
\newcommand{\N}{\rm I\!N}
\newcommand{\universe}{\powerset(\R) - \{\N\}}
\newcommand{\poset}{(U, \subseteq)}


\begin{document}

\begin{frame}{Enunciado}
  \begin{center}
    Sea $U = \universe$. \\
    $\poset$ no es un reticulado.
  \end{center}
\end{frame}

\begin{frame}{Demostración}
  Basta con encontrar dos elementos de $U$ tales que no tengan supremo en $\poset$.
  \stepcounter{beamerpauses}
  \begin{itemize}[<+->]

    \item Sean $C_1 , C_2 \in U$ tales que $C_1 \cup C_2 = \N$
    %% Es trivial que tales C_1 y C_2 existen, por ejemplo C_1 = {1} y C_2 = N_{>1}

    \item Probemos por el absurdo que no existe $sup(\{C_1, C_2\})$
    %% i.e., que no existe ningún elemento de $U$ que sea la mínima cota
    %% superior de $\{C_1, C_2\}$.
  \end{itemize}
\end{frame}

\begin{frame}{Suposición para llegar al absurdo}
    Supongamos $X \in U$ es el supremo de $\{C_1, C_2\}$.
\end{frame}

\begin{frame}{$X \subseteq \N \subseteq X$}
  \begin{itemize}[<+->]

    \item Entonces $C_1 \subseteq X$ y $C_2 \subseteq X$.
    \item Como $C_1 \cup C_2 = \N$ debe ser que \alert{$\N \subseteq X$}.

    \item Por otro lado sea $S_1 = \N \cup \{0\}$ y $S_2 = \N \cup \{-1\}$.
    \item Como $C_1, C_2 \subseteq S_1$, entonces $S_1$ es cota superior de  $\{C_1, C_2\}$
    \item Por el mismo motivo, $S_2$ es cota superior de  $\{C_1, C_2\}$.

    \item Luego, $X \subseteq S_1$ y $X \subseteq S_2$.

    \item Pero entonces: \alert{$X \subseteq (S_1 \cap S_2) = \N$}

  \end{itemize}
\end{frame}

\begin{frame}{Absurdo}
  \begin{itemize}[<+->]

    \item Vimos que $\N \subseteq$ $X$ y que $X$ $\subseteq \N$.
    \item Por antisimetría de $\subseteq$, \alert{$\N = X$}.
    \item Pero {$\N \notin U$}. \alert{Absurdo} pues $X \in U$.
    \item Como consecuencia no existe el supremo de $\{C_1, C_2\}$.
  \end{itemize}
\end{frame}

\begin{frame}[standout]
  \begin{center}
  $(\universe, \subseteq)$ no es un reticulado
  \end{center}
\end{frame}

\end{document}
